\documentclass[]{beamer}

\usepackage{subfiles}
\usepackage{preamble}

\begin{document}

\begin{frame}
\maketitle
\end{frame}

\subfile{dylan.tex}
\subfile{adam.tex}
\subfile{monad_intro.tex}

\section{Contributions and Resources}
\begin{frame}[fragile]
    \begin{itemize}
        \item Author Contributions\\{\scriptsize(in alphabetical order of first name)}
            \begin{itemize}
                \item Adam St. Arnaud:
                    ...
                \item Calvin Beck:
                    List Monad, TicTacToe, IO Monad, Editing
                \item Dylan Ashley:
                    ...
                \item Hannah Barlow:
                    ...
                \item Jed Barlow:
                    ...
            \end{itemize}
        \item Resources
            \begin{itemize}
                \item \url{haskell.org/haskellwiki/Functional_programming}
                \item \url{haskell.org/haskellwiki/Introduction#What_is_functional_programming.3F}
                \item \url{en.wikipedia.org/wiki/Functional_programming}
                \item \url{cse.chalmers.se/~rjmh/Papers/whyfp.pdf}
                \item \url{stackoverflow.com/questions/602444/what--is--functional--declarative--and--imperative--programming}
                \item \url{haskell.cs.yale.edu/wp--content/uploads/2011/01/cs.pdf}
            \end{itemize}
    \end{itemize}
\end{frame}

\begin{frame}[fragile]
  \frametitle{More Resources}

  \begin{itemize}    
  \item \url{wired.com/wiredenterprise/2011/10/john--mccarthy--father--of--ai--and--lisp--dies--at--84}
  \item \url{en.wikipedia.org/wiki/Symbolics}
  \item \url{en.wikipedia.org/wiki/Lisp_machine}
  \item \url{psg.com/~dlamkins/sl/chapter32.html}
  \item \url{http://en.wikibooks.org/wiki/Haskell/Understanding_monads}
  \item \url{http://en.wikibooks.org/wiki/Haskell/Understanding_monads/List}
  \item \url{http://en.wikibooks.org/wiki/Haskell/Understanding_monads/IO}
  \end{itemize}
\end{frame}

\begin{frame}[fragile]
  \frametitle{More Resources}

  \begin{itemize}    
  \item \url{http://en.wikipedia.org/wiki/Tic-tac-toe#Number_of_possible_games}
  \end{itemize}

\end{frame}

\end{document}
