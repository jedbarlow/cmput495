\documentclass[presentation.tex]{subfiles}
\begin{document}


\section{Benefits of Functional Programming}
\begin{frame}
  \frametitle{Referential Transparency}
    	\begin{itemize}
 	\item An expression is referentially transparent if it can be replaced with its value without changing the behaviour of the program.\\
	\item A core principle of functional programming. \\
	\item Allows the programmer to reason about a program's behaviour.
	  	\end{itemize}
\end{frame}

\begin{frame}
  \frametitle{Higher Order Functions}
     	\begin{itemize}
 	\item A higher order function takes one or more functions as input or outputs a function.
	\item Heavily used in functional programming.
	  	\end{itemize} 
\end{frame}

\begin{frame}
  \frametitle{Higher Order Functions}
	 \begin{itemize}
	 \item Example (Python): \\ \;
	
	{def f(x): \\
	\addtolength{\leftskip}{7mm} return x + 3\\}

	{def twice(function, x): \\
	\addtolength{\leftskip}{7mm} return function(function(x)) \\}
	print(twice(f,7)) \;
	
	\item Example (Haskell): \\ \;
	f = (+ 3) \\
	twice function = function . function \\
	main = print (twice f 7)
	\item prints out 13
	  \end{itemize} 
\end{frame}



\begin{frame}
  \frametitle{Lazy Evaluation}
  	\begin{itemize}
\item   An evaluation strategy which delays the evaluation of an expression until its value is needed and which also avoids repeated evaluations. \\
 \item This allows benefits such as programming with infinite lists that are only calculated when called upon.
   \end{itemize}

\end{frame}


\begin{frame}
  \frametitle{Lazy Evaluation}
    	\begin{itemize}
 \item Example: \\ \;
  x = [1..] \\
  take 5 (map ($\wedge$ 2) x) \\
 \item This code creates an "infinite" list of numbers, squares each of them, and then returns a list of the first five. \\
 \item "map" is a higher order function which applies a given function to every element of a given list
\item returns [1 4 9 16 25]
	  \end{itemize} 
\end{frame}  

\section{References}
\begin{frame}
  \frametitle{References}


\end{frame}


\end{document}