\documentclass[presentation.tex]{subfiles}

% An example showing M a --> M b where a != b might be good, e.g. add
% the show function to the maybe monad to go from a Number to a
% String.

\begin{document}

\section{Monads}
\begin{frame}
  \frametitle{What is a Monad?}

  A way of sequencing computations with extra processing being done
  implicitly at each step.

  %% Maibe include what wikipedia says...
  %In functional programming, a monad is a structure that represents
  %computations defined as sequences of steps. A type with a monad
  %structure defines what it means to chain operations, or nest
  %functions of that type together.  \pause
\end{frame}

\begin{frame}
  \frametitle{Formal Definition}

  A Monad is a collection of the following entities.

  \begin{description}
  \item[Type Constructor]
    $M\ a$

  \item[Unit Function]
    $f::a\rightarrow M\ a$\\
    Wrap a regular value into a Monadic type.

  \item[Bind Function]
    $f::M\ a\rightarrow(a\rightarrow M\ b)\rightarrow b$\\
    Unwrap $M\ a$, do some processing and get a value of type $b$, then
    apply $(a\rightarrow M\ b)$ to get a new wrapped value, maybe do
    more processing, return a value of type $M\ b$.

    Note that $a=b$ may or may not be the case.
  \end{description}

  Satisfying the following conditions...
\end{frame}

\begin{frame}
  \frametitle{Monad Rules}
  % Add rules that have to be satisfied
\end{frame}

\begin{frame}
  \frametitle{How Monads Are Used}

  Sequence together ``computation'' functions of type $a \rightarrow
  M\ b$ using the $bind$ function.
\end{frame}
\end{document}
