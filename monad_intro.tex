\documentclass[presentation.tex]{subfiles}

% An example showing M a --> M b where a != b might be good, e.g. add
% the show function to the maybe monad to go from a Number to a
% String.

\begin{document}

\section{Monads}
\begin{frame}
  \frametitle{What is a Monad?}

  A way of sequencing computations with extra processing being done
  implicitly at each step.

  %This extra processing means that the behavior of computation can be
  %based on earlier results.

  %% Maibe include what wikipedia says...
  %In functional programming, a monad is a structure that represents
  %computations defined as sequences of steps. A type with a monad
  %structure defines what it means to chain operations, or nest
  %functions of that type together.  \pause
\end{frame}

\begin{frame}
  \frametitle{Formal Definition}

  A Monad is a collection of the following entities.

  \begin{description}
  \item[Type Constructor]
    $M\ a$

  \item[Unit Function]
    $unit::a\rightarrow M\ a$\\
    Wrap a regular value into a Monadic type.

  \item[Bind Function]
    $bind::M\ a\rightarrow(a\rightarrow M\ b)\rightarrow b$\\
    Unwrap $M\ a$, do some processing and get a value of type $b$,
    then apply $(a\rightarrow M\ b)$ to get a new wrapped value,
    return a value of type $M\ b$.

    Note that $a=b$ may or may not be the case.
  \end{description}

  Satisfying the following conditions...
\end{frame}

\begin{frame}
  \frametitle{Monad Rules}
  % Add rules that have to be satisfied
  For any values $x::a$, $m::M\ b$ and any functions $f::a \rightarrow M\ b$, $g::b \rightarrow M\ c$, $h::c \rightarrow M\ d$,
  \begin{itemize}
  \item
    $bind(unit(x), f) = f(x)$
  \item
    $bind(m, unit) = m$
  \item
    $bind(bind(m, g), h) = bind(m, (\lambda y \rightarrow bind(g(y), h)))$
  \end{itemize}
\end{frame}

\begin{frame}
  \frametitle{Haskell Syntax}
  $a >>= b = bind(a, b)$
\end{frame}

\begin{frame}
  \frametitle{Monad Rules}
  % Add rules that have to be satisfied
  For any values $x::a$, $m::M\ d$ and any functions $f::a \rightarrow M\ b$, $g::b \rightarrow M\ c$, $h::c \rightarrow m\ d$,
  \begin{itemize}
  \item
    $unit(x) >>= f = f(x)$
  \item
    $m >>= unit = m$
  \item
    $(m >>= g) >>= h = m >>= (\lambda y \rightarrow g(y) >>= h)$
  \end{itemize}
\end{frame}

\begin{frame}
  \frametitle{How Monads Are Used}

  Sequence together ``computation'' functions of type $a \rightarrow
  M\ b$ using the $bind$ function.
\end{frame}
\end{document}
