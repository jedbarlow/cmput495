\documentclass[presentation.tex]{subfiles}

% An example showing M a --> M b where a != b might be good, e.g. add
% the show function to the maybe monad to go from a Number to a
% String.

\begin{document}

\section{Monads}
\begin{frame}
  \frametitle{Problems!}

  With purity and laziness some problems come up.

  \begin{itemize}
  \item How can we do things like I/O if functions always return the same value?
  \item What about randomness?
  \item How do we make sure laziness doesn't interfere with things like I/O?
  \end{itemize}
\end{frame}

\begin{frame}
  \frametitle{What is a Monad?}

  A way of sequencing computations with extra processing being done
  implicitly at each step.

  %This extra processing means that the behavior of computation can be
  %based on earlier results.

  %% Maibe include what wikipedia says...
  %In functional programming, a monad is a structure that represents
  %computations defined as sequences of steps. A type with a monad
  %structure defines what it means to chain operations, or nest
  %functions of that type together.  \pause
\end{frame}

\begin{frame}
  In other words, monads are just a generalization of the semicolon in
  languages like C. They're a way of sequencing operations while doing
  the most likely thing you need to do in between the steps.

  They're a nice mathematical abstraction of ``imperative languages''.
\end{frame}

\begin{frame}
  \frametitle{Formal Definition}

  A Monad is a collection of the following entities.

  \begin{description}
  \item[Type Constructor]
    $M\ a$

  \item[Unit Function]
    $\return::a\rightarrow M\ a$\\
    Wrap a regular value into a Monadic type.

  \item[Bind Function]
    $(>>=)::M\ a\rightarrow(a\rightarrow M\ b)\rightarrow M\ b$\\
    Unwrap $M\ a$, do some processing and maybe get a value of type $a$, then
    apply $(a\rightarrow M\ b)$ to get a new wrapped value, return a value of
    type $M\ b$.

    Note that $a=b$ may or may not be the case.
  \end{description}
\end{frame}

\begin{frame}
  \frametitle{How Monads Are Used}

  Sequence together ``computation'' functions of type $a \rightarrow
  M\ b$ using the $>>=$ function.
\end{frame}

\begin{frame}
  \frametitle{Monad Rules}
  For any values $x::a$, $m::M\ d$ and any functions $f::a \rightarrow M\ b$, $g::b \rightarrow M\ c$, $h::c \rightarrow m\ d$,
  \begin{itemize}
  \item
    $\return x >>= f = f x$
  \item
    $m >>= \return = m$
  \item
    $(m >>= g) >>= h = m >>= (\lambda y \rightarrow g y >>= h)$
  \end{itemize}

  The monad laws are extremely valuable for helping the programmer
  understand what monadic code is doing. It provides guarantees about
  the behaviour of bind ($>>=$) and unit (return)
\end{frame}

% Python example?

\begin{frame}
  \frametitle{List Monad}

  The list monad represents a non-deterministic computation. Any
  function which returns a list can be viewed as a non-deterministic
  computation -- each element in the list is a possible result.

  Tic Tac Toe can be modeled this way. Each turn produces a number of
  possible results depending on where the player decides to go.
\end{frame}

\begin{frame}[fragile]
  \frametitle{List Monad}

  Getting a list of all possible states after two turns is a bit more
  difficult! With the list monad we can sequence turns quite easily!

  \begin{lstlisting}[frame=single,language=Haskell,breaklines=true]
    -- Definition of return
    return x = [x]

    -- Definition of bind
    xs >>= f = concat (map f xs)
  \end{lstlisting}

  Now let's look at Tic Tac Toe.
\end{frame}

\begin{frame}[fragile]
  \frametitle{List Monad Laws}

  Recall the Monad laws...

  \begin{itemize}
  \item
    $\return x >>= f = f x$
  \item
    $m >>= \return = m$
  \item
    $(m >>= g) >>= h = m >>= (\lambda y \rightarrow g y >>= h)$
  \end{itemize}

  It's clear that the list monad obeys these laws.
\end{frame}

\begin{frame}[fragile]
  \frametitle{IO Monad}

  IO is pretty tricky business in a pure / lazy language.

  \begin{itemize}
  \item IO involves external state
  \item IO functions can not be referentially transparent.

  \begin{lstlisting}[frame=single,language=Haskell,breaklines=true]
    -- Clearly not a guarantee!
    getChar == getChar
  \end{lstlisting}

  \item No guarantee of execution order because of laziness.
  \end{itemize}

  However we clearly need IO, otherwise we can't really do anything.
\end{frame}

\begin{frame}[fragile]
  \frametitle{IO Monad}

  Any function which performs any IO can not be guaranteed to be
  referentially transparent. We want to minimize the number of IO
  functions, and we want to keep track of functions that perform IO.
\end{frame}

\begin{frame}[fragile]
  \frametitle{IO Monad}

  Haskell provides the IO monad. When wrapped around a type you can
  not retrieve any values from the IO monad -- the value is stuck
  within the IO type forever.

  This causes anything which performs IO to have an IO type, so it is
  appropriately labeled as performing IO. IO monad taints any function
  which uses it, making sure that it is typed as being impure.
\end{frame}

\begin{frame}[fragile]
  \frametitle{Elmer Fudd The Glorious IO Example}

  So, it seems we can't mix IO at all with pure functions, which means
  poor Elmer Fudd will never talk to the outside world.

  \begin{lstlisting}[frame=single,language=Haskell,breaklines=true]
    -- Turns any string into soft spoken Fuddy fun times
    elmerFudd :: String -> String
    elmerFudd = map (\c -> if c == 'r' || c == 'R' then 'w' else toLower c)
  \end{lstlisting}

  This is a simple enough pure function, but how can we actually give
  Elmer their script?
\end{frame}

\begin{frame}[fragile]
  \frametitle{Elmer Fudd The Glorious IO Example}

  Monads are a way of sneakily unwrapping IO. Look at bind:

  \[(>>=)::M\ a\rightarrow(a\rightarrow M\ b)\rightarrow M\ b\]

  In the context of the IO monad this becomes:

  \[(>>=)::\io\ a\rightarrow(a\rightarrow \io\ b)\rightarrow \io\ b\]

  We can cheese the value in the first piece of IO out, as long as the
  function takes the value to another value mapped in IO.
\end{frame}

\end{document}
