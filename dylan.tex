\documentclass[presentation.tex]{subfiles}

\begin{document}

\large
\section{Definition}
Functional Programming can be best summarized as creating programs that consist solely of functions.

\vspace{\baselineskip}
A program written in a functional language can be though of as a tree with a single main function which subsequently calls other functions, which call other functions, etcetera.

\vspace{\baselineskip}
Functions consist of input, a manipulation of that input, and an output. In functional programming the same input should always lead to the same output. This means there is predictability!

\vspace{\baselineskip}
The most shocking thing about functional languages is that a variable, once given a value, will never change. You can think of variables in a functional language like a variable in mathematics. % Reinforces the paradigm of same input yields same output.

% Functions are first class citizens.

\pagebreak
\section{History}
\subsection{A stack based language: Lisp}
Lisp was based on a simple idea: make everything a pair.

\vspace{\baselineskip}
Adding 2 and 3 together in this ideology would actually be carried out as (addition 2 3). If we wanted to compute a more complicated operation we would simply have a deeper stack: (multiply 2 (multiply 3 4)).

\pagebreak
\section{Motivation}
A functional program is often much shorter than a imperative program. This also helps to make them often much easier to understand and write without errors.

\vspace{\baselineskip}
Quicksort written in haskell:
\scriptsize
\begin{verbatim}
quicksort :: Ord a => [a] -> [a]
quicksort [] = []}
quicksort (p:xs) = (quicksort lesser) ++ [p] ++ (quicksort greater)
    where
      lesser = filter (< p) xs
      greater = filter (>= p) xs
\end{verbatim}
\large

\vspace{\baselineskip}
This quicksort will sort anything where sort could be defined on. There is no need to specify a type or a size.
\vspace{\baselineskip}
An even shorter quicksort, notice the set builder notation:
\scriptsize
\begin{verbatim}
qsort (p:xs) = qsort [x | x<-xs, x<p] ++ [p] ++ qsort [x | x<-xs, x>=p]
\end{verbatim}
\large

\end{document}